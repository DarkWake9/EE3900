\documentclass[journal,12pt,twocolumn]{IEEEtran}

\usepackage{setspace}
\usepackage{gensymb}
\singlespacing
\usepackage[cmex10]{amsmath}

\usepackage{amsthm}

\usepackage{mathrsfs}
\usepackage{txfonts}
\usepackage{stfloats}
\usepackage{bm}
\usepackage{cite}
\usepackage{cases}
\usepackage{subfig}

\usepackage{longtable}
\usepackage{multirow}

\usepackage[utf8]{inputenc}
\usepackage[T1]{fontenc}
\usepackage{amsmath}
\usepackage{amsfonts}
\usepackage{amssymb}
\usepackage{mhchem}
\usepackage{stmaryrd}
\usepackage{graphicx}
\usepackage[export]{adjustbox}

\usepackage{enumitem}
\usepackage{mathtools}
\usepackage{steinmetz}
\usepackage{tikz}
\usepackage{circuitikz}
\usepackage{verbatim}
\usepackage{tfrupee}
\usepackage[breaklinks=true]{hyperref}
\usepackage{graphicx}
\usepackage{tkz-euclide}

\usetikzlibrary{calc,math}
\usepackage{listings}
    \usepackage{color}                                            %%
    \usepackage{array}                                            %%
    \usepackage{longtable}                                        %%
    \usepackage{calc}                                             %%
    \usepackage{multirow}                                         %%
    \usepackage{hhline}                                           %%
    \usepackage{ifthen}                                           %%
    \usepackage{lscape}     
\usepackage{multicol}
\usepackage{chngcntr}

\DeclareMathOperator*{\Res}{Res}

\renewcommand\thesection{\arabic{section}}
\renewcommand\thesubsection{\thesection.\arabic{subsection}}
\renewcommand\thesubsubsection{\thesubsection.\arabic{subsubsection}}

\renewcommand\thesectiondis{\arabic{section}}
\renewcommand\thesubsectiondis{\thesectiondis.\arabic{subsection}}
\renewcommand\thesubsubsectiondis{\thesubsectiondis.\arabic{subsubsection}}
\providecommand{\nCr}[2]{\,^{#1}C_{#2}} % nCr
\providecommand{\nPr}[2]{\,^{#1}P_{#2}} % nPr
\providecommand{\mbf}{\mathbf}
\providecommand{\pr}[1]{\ensuremath{\Pr\left(#1\right)}}
\providecommand{\qfunc}[1]{\ensuremath{Q\left(#1\right)}}
\providecommand{\sbrak}[1]{\ensuremath{{}\left[#1\right]}}
\providecommand{\lsbrak}[1]{\ensuremath{{}\left[#1\right.}}
\providecommand{\rsbrak}[1]{\ensuremath{{}\left.#1\right]}}
\providecommand{\brak}[1]{\ensuremath{\left(#1\right)}}
\providecommand{\lbrak}[1]{\ensuremath{\left(#1\right.}}
\providecommand{\rbrak}[1]{\ensuremath{\left.#1\right)}}
\providecommand{\cbrak}[1]{\ensuremath{\left\{#1\right\}}}
\providecommand{\lcbrak}[1]{\ensuremath{\left\{#1\right.}}
\providecommand{\rcbrak}[1]{\ensuremath{\left.#1\right\}}}
\theoremstyle{remark}
\newtheorem{rem}{Remark}
\newcommand{\sgn}{\mathop{\mathrm{sgn}}}
\providecommand{\abs}[1]{\left\vert#1\right\vert}
\providecommand{\res}[1]{\Res\displaylimits_{#1}} 
\providecommand{\norm}[1]{\lVert#1\rVert}
\providecommand{\mtx}[1]{\mathbf{#1}}
\providecommand{\mean}[1]{E\left[ #1 \right]}
\providecommand{\fourier}{\overset{\mathcal{F}}{ \rightleftharpoons}}
\providecommand{\ztrans}{\overset{\mathcal{Z}}{ \rightleftharpoons}}
%\providecommand{\hilbert}{\overset{\mathcal{H}}{ \rightleftharpoons}}
\providecommand{\system}{\overset{\mathcal{H}}{ \longleftrightarrow}}
	%\newcommand{\solution}[2]{\textbf{Solution:}{#1}}
\newcommand{\solution}{\noindent \textbf{Solution: }}
\providecommand{\dec}[2]{\ensuremath{\overset{#1}{\underset{#2}{\gtrless}}}}
\numberwithin{equation}{section}
%\numberwithin{equation}{subsection}
%\numberwithin{problem}{subsection}
%\numberwithin{definition}{subsection}
\makeatletter
\@addtoreset{figure}{problem}
\makeatother
\let\StandardTheFigure\thefigure
%\renewcommand{\thefigure}{\theproblem.\arabic{figure}}
\renewcommand{\thefigure}{\theproblem}


\hyphenation{op-tical net-works semi-conduc-tor}
\def\inputGnumericTable{}                                 %%

\lstset{
%language=C,
frame=single, 
breaklines=true,
columns=fullflexible
}
\begin{document}
\title{Assignment 1}
\author{Vibhavasu Pasumarti - EP20BTECH11015}
\maketitle
\newpage
\bigskip
\renewcommand{\thefigure}{\theenumi}
\renewcommand{\thetable}{\theenumi}
Download all python codes from 
\begin{lstlisting}
https://github.com/VIB2020/AI1103/blob/main/Assignment%201/code/Assignment_1.py
\end{lstlisting}
%
and latex-tikz codes from 
%
\begin{lstlisting}
https://github.com/VIB2020/AI1103/blob/main/Assignment%201/Assignment%201.tex
\end{lstlisting}

%\section{Difference Equation}
\section{}
\section{}
\section{}
\section{Z - transform}
\subsection{}
\begin{enumerate}[label=\roman*)]
    \item  To prove:    ${\mathcal {Z}}\{x(n-1)\} = z^{-1}X(z)$\\

    \item To find:  ${\mathcal {Z}}\{x(n-k)\}$
\end{enumerate}

\begin{enumerate}[label=\roman*)]
   \item \begin{align}
        \mathcal{Z} \{x(n-1)\} = \sum_{n=-\infty}^{\infty} x(n-1) z^{-n}\\
        \intertext{put $n = m + 1$}
        \implies \sum_{m=-\infty}^{\infty} x[m] z^{-m-1}\\
        =z^{-1} \sum_{m=-\infty}^{\infty} x[m] z^{-m}
    \\
        =z^{-1} $X$(z)\\
     \end{align}

    \item Similarly 
    \begin{equation}
        \mathcal{Z}\{x[n-k]\}=\sum_{n=-\infty}^{\infty} x[n-k] z^{-n}
    \end{equation}    
        \intertext{Let $m=n-k$}
    \begin{align}
        =\sum_{m=-\infty}^{\infty} x[m] z^{-m-k}\\
        =z^{-k} X[z]
    \end{align}

\end{enumerate}
\subsection{To prove:}
\begin{align}
    H(z)=& \frac{Y(z)}{X(z)}    
\end{equation}
(Assuming  Z-transform is linear)
\begin{align}
    y(n)+\frac{y(n-1)}{2}=x(n)+x(n-z)
    \intertext{k y(n)=0, n<2}
    \implies y(z)+\frac{z^{-1} y(z)}{2}=x(z)+z^{-2} x(z) \\
    H(z)=\frac{1+z^{-2}}{1-z^{-1}}
\end{align}

\subsection{}
\begin{enumerate}[label=(\roman*)]
    \item Find $Z$-transform of $\delta(n)= \begin{cases}1 & n=0 \\ 0 & n \neq 0\end{cases}$

    \item Show $Z$-transform of $\delta(n)= \begin{cases}1 & n \geqslant 0 \\ 0 & n<0\end{cases}$ is:
\end{enumerate}

\begin{enumerate}[label=(\roman*)]
    \item) \solution \begin{equation}
        \Delta(z) = z\{\delta[n]\}=\sum_{n=-\infty}^{\infty} \delta(n) z^{-n} = 1\\
    \end{equation}

    \item) \solution\begin{align}
        U(z) = Z\{\delta(n)\}=\sum_{n=-\infty}^{\infty} u[n] z^{-n} \\
        =1+z^{-1}+z^{-2}+\cdots \\
        =\frac{1}{1-z^{-1}}
    \end{align}
\end{enumerate}
\subsection{}
    \begin{align}
        a^{n} u[n]  \rightleftharpoons ?|z|>|a| \\
        u[n] = \begin{cases}1 & n \geq 0 \\
            0 & n<0\end{cases} \\
        U^{\prime}(z) &=z\{u[n]\}=\sum_{n=-\infty}^{\infty} a^{n}[n] z^{-n} \\
        &=\frac{1}{1-a z^{-1}}
    \end{align}
    
\section{Impulse Response}
\subsection{}
Find $h[n] \underset{z}{\rightleftharpoons} H(z)$

Given: $\exists$ 1-1 relationship b/w $h[n]$ \& $H(z)$

$h(n)$ is the Impulse Response of the system defined in $3.2$
$y(n)+\frac{y[n-1]}{2}=x[n]+x[n-2]$
From $4 \cdot 2 \quad H(z)=\frac{1+z^{-2}}{1-\frac{z^{-1}}{2}}$\\
$\sum_{n=-\infty}^{\infty} h[n] z^{-n}=\frac{1+z^{-2}}{1-\frac{z^{-1}}{2}}=\frac{Y(z)}{x(z)}$

From $4 \cdot 4$ :
$$
\begin{align}
a^{n} u[n] & \frac{Z}{\bar{v}} v^{\prime}(z)=\frac{1}{1-a z^{-1}} \\
1+(z) &=\frac{1}{1-\frac{z^{-1}}{2}}+\frac{z^{-2}}{1-\frac{z^{-1}}{2}} \\
&=z\left\{\frac{z}{z} \frac{u[n]}{2^{n}}\right\}+z^{-2} z\{u[n]\} \\
=& n[n]=\frac{u[n]}{z^{n}}+\frac{u[n-2]}{2^{n-2}} .
\end{align}
$$

$$
\sum_{n=-\infty}^{\infty} h(n)<\infty
$$
\subsection{}
\subsection{}
5.3 1s the system defined in $3.2$
$$
\rightarrow y(n)-\frac{y(n-1]}{2}=x[n]+x[n-2]
$$
stable? For the impulse response $h(n)$ in 5.1?

from $5 \cdot 1: h[n]=\frac{u[n]}{2^{n}}+\frac{u[n-2]}{2^{n-2}}$
$$
\sum_{n=-\infty}^{\infty} h[n]=\sum_{n=0}^{\infty} \frac{u[n]}{2^{n}}+\sum_{n=2}^{\infty} \frac{u[n-2]}{2^{n-2}}
$$
$=\left(1+\frac{1}{2}+\cdots \cdot\right) 2$

$=4<\infty$

$\Rightarrow h(n)$ is STABLE

\subsection{}
\begin{aligned}
 h(n)-\frac{h(n-1)}{2}=\delta(n)+\delta(n-2) \\
 h[n]=? \\
 S(n)= \begin{cases}1 & n=0 \\ 0 & n \neq 0\end{cases} \\
 h=0 \rightarrow h(0)-\frac{h(-1)}{2}=1 \\
 n=1 \rightarrow h(1)-\frac{h(0)}{2}=0 \Rightarrow h(1)=2 h(0) \\
 n=2 \quad n(2)-\frac{n(1)}{2}=1 \\
 \Rightarrow h(2)-h(0)=1 \\
 \text { Assume } h[n]=0 \quad \forall n<0 \\
 h(0)=1 \quad n(1)=2 \quad n(2)=2 \\
 h(3)=\frac{h(2)}{2}=1 \quad h(4)=\frac{h(3)}{2}=\frac{1}{2} \\
 \Rightarrow n[n]= \begin{cases}1 & n=0 \\ 2 & n=1 \\ \frac{1}{2^{n-3}} & n \geqslant 2\end{cases} 
\end{aligned}

\subsection{}

\begin{equation}
y(n)=x(n) & * h[n]=\sum_{n=-\infty}^{\infty} x(k) h(n-k) \\
\end{equation}
$$
\subsection{} To prove: \quad
y(n)=\sum_{n=-\infty}^{\infty} x(n-k) h(k)$ \\
6.2 To prove: $Y(k)=x(k)_{\infty} 1+(k)$
\begin{align}
 X(z)=z\{x(n)\}=\sum_{n=-\infty}^{\infty} x(n) z^{-n} \\
 1+(z)=Z\left\{(h(m)\}=\sum_{m=-\infty}^{\infty} h(m) z^{-m}\right. \\
 X(z) \cdot H(z)=\sum_{n=-\infty}^{\infty} x(n) z^{-n} \sum_{m=-\infty}^{\infty} h(m) z^{-m} \\
 =\sum_{n=-\infty}^{\infty} \sum_{m=-\infty}^{\infty} x[n] h[m] z^{-(n+m)} \\
 m=k-n \quad m=-\infty \rightarrow n=\infty \quad m=\infty \rightarrow n=-\infty \\
 =\sum_{k=-\infty}^{\infty}\left(\sum_{n=-\infty}^{\infty} x[n] h[n-k]\right) z^{-k} \\
 =\sum_{k=-\infty}^{\infty} y[n] z^{-k}=Y(k) \\
 \Rightarrow \quad Y(k)=1+(k) \cdot k(k) 
\end{align}

From 6.2:

\begin{align}
Y(z)=x(z) \cdot H(z)=\sum_{n=-\infty}^{\infty} x(n) \sum_{m=-\infty}^{\infty} h(m) z^{-(m+n)} \\
m=n-k \rightarrow Y(z)=\sum_{n=-\infty}^{\infty}\left(\sum_{n=-\infty}^{\infty} x(n) h(n-k) z^{-k}\right. \\
y^{1}(n)
\end{align}

now pst $n+m=k \quad n=-\infty$

\begin{aligned}
& \Rightarrow Y(z)=\sum_{k=-\infty}^{\infty} x(m-k) \sum_{m=-\infty}^{\infty} h(m) z^{-k} \\
& =\sum_{k=-\infty}^{\infty}\left(\sum_{m=-\infty}^{\infty} x[m-k] h[k]\right) z^{-k} \\
& \text { but } Y(z)=\sum_{k=-\infty}^{\infty} y(m) z^{-k} \\
& \Rightarrow y(m)=\sum_{m=-\infty}^{\infty} x[m-k] h(k) \\
& 5 \cdot 6 \rightarrow y(n)=\sum_{n=-\infty}^{\infty} x(n-k) h(k) 
\end{aligned}
$$
\section{DFT and FFT}
\subsection{} Compote $x(k) \triangleq \sum_{n=0}^{N-1} x(n) \exp \left(\frac{-i 2 \pi k n}{N}\right)$ $k=0,1 \cdots \cdots-1$

\& $H(k)$ using $h(n)$
$$
x_{(n)}=\{1,2,3,4,3,1\} \quad N=6
$$
$x_{(k)}=\sum_{n=0}^{5} x_{(n)} \exp \left(\frac{-i 2 \pi k n}{6}\right)$

$x_{(k)}=1+2 \exp \left(-\frac{i \pi_{k}}{3}\right)+3 \exp \left(-\frac{i 2 \pi_{k}}{3}\right)$

$+4 \exp (-i \pi k)+2 \exp \left(-i \frac{4 \pi k}{3}\right)$

$+\exp \left(-\frac{i 5 \pi k}{6}\right)$

$k=0,1,2,3,4,5$

$X(0)=13$

$\begin{aligned} x(1)=1+2 e^{-\frac{i \pi}{3}}+3 e^{-\left(\frac{2 \pi i}{3}\right)}+&+e^{-2 \pi}+2 e^{-\left(\frac{4 \pi i}{3}\right)} \\ &+e^{-\frac{5 \pi i}{6}} \end{aligned}$

$=$ ii) $H(k)$
$$
\begin{aligned}
&\text { (k) with } h(n)= \begin{cases}1 & n=0 \\
2 & n=1 \\
\frac{1}{2^{n-3}} & n \geqslant 2 \\
& l N \rightarrow \infty\end{cases} \\
&f(k)=\sum_{n=0}^{N-1} h(n) \exp \left(-i 2 \pi k \frac{k}{n}\right)
\end{aligned}

$$
x_{(n)}=\{1,2,3,4,3,1\} \quad N=6
$$
$x_{(k)}=\sum_{n=0}^{5} x_{(n)} \exp \left(\frac{-i 2 \pi k n}{6}\right)$

$x_{(k)}=1+2 \exp \left(-\frac{i \pi_{k}}{3}\right)+3 \exp \left(-\frac{i 2 \pi_{k}}{3}\right)$

$+4 \exp (-i \pi k)+2 \exp \left(-i \frac{4 \pi k}{3}\right)$

$+\exp \left(-\frac{i 5 \pi k}{6}\right)$

$k=0,1,2,3,4,5$

$X(0)=13$

$\begin{aligned} x(1)=1+2 e^{-\frac{i \pi}{3}}+3 e^{-\left(\frac{2 \pi i}{3}\right)}+&+e^{-2 \pi}+2 e^{-\left(\frac{4 \pi i}{3}\right)} \\ &+e^{-\frac{5 \pi i}{6}} \end{aligned}$

$=$ ii) $H(k)$
$$
\begin{aligned}
&\text { (k) with } h(n)= \begin{cases}1 & n=0 \\
2 & n=1 \\
\frac{1}{2^{n-3}} & n \geqslant 2 \\
& l N \rightarrow \infty\end{cases} \\
&f(k)=\sum_{n=0}^{N-1} h(n) \exp \left(-i 2 \pi k \frac{k}{n}\right)
\end{aligned}
$$
\section{Exercises}
\end{document}