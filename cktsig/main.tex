\documentclass[journal,12pt,twocolumn]{IEEEtran}
%
\usepackage{setspace}
\usepackage{gensymb}
\usepackage{xcolor}
\usepackage{caption}
%\usepackage{subcaption}
%\doublespacing
\singlespacing
\usepackage{siunitx}
%\usepackage{graphicx}
%\usepackage{amssymb}
%\usepackage{relsize}
\usepackage[cmex10]{amsmath}
\usepackage[thinc]{esdiff}
\usepackage{mathtools}
%\usepackage{amsthm}
%\interdisplaylinepenalty=2500
%\savesymbol{iint}
%\usepackage{txfonts}
%\restoresymbol{TXF}{iint}
%\usepackage{wasysym}
\usepackage{hyperref}
\usepackage{amsthm}
\usepackage{mathrsfs}
\usepackage{txfonts}
\usepackage{stfloats}
\usepackage{cite}
\usepackage{cases}
\usepackage{subfig}
%\usepackage{xtab}
\usepackage{longtable}
\usepackage{multirow}
%\usepackage{algorithm}
%\usepackage{algpseudocode}
%\usepackage{enumerate}
\usepackage{enumitem}
\usepackage{mathtools}
%\usepackage{iithtlc}
%\usepackage[framemethod=tikz]{mdframed}
\usepackage{listings}
\usepackage{tikz}
\usetikzlibrary{shapes,arrows,positioning}
\usepackage{circuitikz}
\let\vec\mathbf


%\usepackage{stmaryrd}


%\usepackage{wasysym}
%\newcounter{MYtempeqncnt}
\DeclareMathOperator*{\Res}{Res}
%\renewcommand{\baselinestretch}{2}
\renewcommand\thesection{\arabic{section}}
\renewcommand\thesubsection{\thesection.\arabic{subsection}}
\renewcommand\thesubsubsection{\thesubsection.\arabic{subsubsection}}

\renewcommand\thesectiondis{\arabic{section}}
\renewcommand\thesubsectiondis{\thesectiondis.\arabic{subsection}}
\renewcommand\thesubsubsectiondis{\thesubsectiondis.\arabic{subsubsection}}

%\renewcommand{\labelenumi}{\textbf{\theenumi}}
%\renewcommand{\theenumi}{P.\arabic{enumi}}

% correct bad hyphenation here
\hyphenation{op-tical net-works semi-conduc-tor}

\lstset{
	language=Python,
	frame=single, 
	breaklines=true,
	columns=fullflexible
}



\begin{document}
	%
	
	\theoremstyle{definition}
	\newtheorem{theorem}{Theorem}[section]
	\newtheorem{problem}{Problem}
	\newtheorem{proposition}{Proposition}[section]
	\newtheorem{lemma}{Lemma}[section]
	\newtheorem{corollary}[theorem]{Corollary}
	\newtheorem{example}{Example}[section]
	\newtheorem{definition}{Definition}[section]
	%\newtheorem{algorithm}{Algorithm}[section]
	%\newtheorem{cor}{Corollary}
	\newcommand{\BEQA}{\begin{eqnarray}}
		\newcommand{\EEQA}{\end{eqnarray}}
	\newcommand{\define}{\stackrel{\triangle}{=}}
	\newcommand{\myvec}[1]{\ensuremath{\begin{pmatrix}#1\end{pmatrix}}}
	\newcommand{\mydet}[1]{\ensuremath{\begin{vmatrix}#1\end{vmatrix}}}
	
	\bibliographystyle{IEEEtran}
	%\bibliographystyle{ieeetr}
	
	\providecommand{\nCr}[2]{\,^{#1}C_{#2}} % nCr
	\providecommand{\nPr}[2]{\,^{#1}P_{#2}} % nPr
	\providecommand{\mbf}{\mathbf}
	\providecommand{\pr}[1]{\ensuremath{\Pr\left(#1\right)}}
	\providecommand{\qfunc}[1]{\ensuremath{Q\left(#1\right)}}
	\providecommand{\sbrak}[1]{\ensuremath{{}\left[#1\right]}}
	\providecommand{\lsbrak}[1]{\ensuremath{{}\left[#1\right.}}
	\providecommand{\rsbrak}[1]{\ensuremath{{}\left.#1\right]}}
	\providecommand{\brak}[1]{\ensuremath{\left(#1\right)}}
	\providecommand{\lbrak}[1]{\ensuremath{\left(#1\right.}}
	\providecommand{\rbrak}[1]{\ensuremath{\left.#1\right)}}
	\providecommand{\cbrak}[1]{\ensuremath{\left\{#1\right\}}}
	\providecommand{\lcbrak}[1]{\ensuremath{\left\{#1\right.}}
	\providecommand{\rcbrak}[1]{\ensuremath{\left.#1\right\}}}
	\theoremstyle{remark}
	\newtheorem{rem}{Remark}
	\newcommand{\sgn}{\mathop{\mathrm{sgn}}}
	\providecommand{\abs}[1]{\left\vert#1\right\vert}
	\providecommand{\res}[1]{\Res\displaylimits_{#1}} 
	\providecommand{\norm}[1]{\lVert#1\rVert}
	\providecommand{\mtx}[1]{\mathbf{#1}}
	\providecommand{\mean}[1]{E\left[ #1 \right]}
	\providecommand{\fourier}{\overset{\mathcal{F}}{ \rightleftharpoons}}
	\providecommand{\ztrans}{\overset{\mathcal{Z}}{ \rightleftharpoons}}
	
	%\providecommand{\hilbert}{\overset{\mathcal{H}}{ \rightleftharpoons}}
	\providecommand{\system}[1]{\overset{\mathcal{#1}}{\longleftrightarrow}}
	%\newcommand{\solution}[2]{\textbf{Solution:}{#1}}
	\newcommand{\solution}{\noindent \textbf{Solution: }}
	\providecommand{\dec}[2]{\ensuremath{\overset{#1}{\underset{#2}{\gtrless}}}}
	\numberwithin{equation}{section}
	%\numberwithin{equation}{subsection}
	%\numberwithin{problem}{subsection}
	%\numberwithin{definition}{subsection}
	\makeatletter
	\@addtoreset{figure}{problem}
	\makeatother
	
	\let\StandardTheFigure\thefigure
	%\renewcommand{\thefigure}{\theproblem.\arabic{figure}}
	\renewcommand{\thefigure}{\theproblem}
	\renewcommand{\thefigure}{\arabic{section}.\arabic{figure}}
	\makeatletter
	\@addtoreset{figure}{section}
	\makeatother
	
	%\numberwithin{figure}{subsection}
	
	\def\putbox#1#2#3{\makebox[0in][l]{\makebox[#1][l]{}\raisebox{\baselineskip}[0in][0in]{\raisebox{#2}[0in][0in]{#3}}}}
	\def\rightbox#1{\makebox[0in][r]{#1}}
	\def\centbox#1{\makebox[0in]{#1}}
	\def\topbox#1{\raisebox{-\baselineskip}[0in][0in]{#1}}
	\def\midbox#1{\raisebox{-0.5\baselineskip}[0in][0in]{#1}}
	
	\vspace{3cm}

	\vspace{3cm}
	
	\title{ 
		%\logo{
			%}
		Circuits and Transforms
		%	\logo{Octave for Math Computing }
	}
	\author{VIBHAVASU}


% make the title area
\maketitle

%\newpage

\tableofcontents

%\renewcommand{\thefigure}{\thesection.\theenumi}
%\renewcommand{\thetable}{\thesection.\theenumi}

\renewcommand{\thefigure}{\theenumi}
\renewcommand{\thetable}{\theenumi}

%\renewcommand{\theequation}{\thesection}


\bigskip

\begin{abstract}
This manual provides a simple introduction to Transforms
\end{abstract}



\section{Definitions}
\begin{enumerate}[label=\arabic*.,ref=\thesection.\theenumi]
\numberwithin{equation}{section}
\numberwithin{figure}{section}
\item The unit step function is 
\begin{align}
	u(t) =
	\begin{cases}
		1 & t > 0
		\\
		\frac{1}{2} & t = 0
		\\
		0 & t < 0
	\end{cases}
\end{align}
\item The Laplace transform of $g(t)$ is defined as 
\begin{align}
	G(s) = \int_{-\infty}^{\infty} g(t) e^{-st}\, dt
\end{align}
\end{enumerate}

\section{Laplace Transform}
\begin{enumerate}[label=\arabic*.,ref=\thesection.\theenumi]
\numberwithin{equation}{section}
\item In the circuit, the switch S is connected to position P for a long time so that the charge on the capacitor
becomes $q_1 \, \mu C$. Then S is switched to position Q.  After a long time, the charge on the capacitor is
$q_2 \, \mu C$.
\begin{figure}[!ht]
	\centering
	\includegraphics[width=\columnwidth]{/media/darkwake/VIB2/EE3900/cktsig/figs/ckt.jpg}
	\caption{}
	\label{fig:ckt}
\end{figure}
\item Draw the circuit using latex-tikz.\\
\solution
\begin{figure}[!h]
	\begin{circuitikz} 
		\draw 
		(0,0) to[battery1, l=1 $V$, invert] (0,2)
		-- (0.5,2) node[label={above:P}] {}
		to[R, l^=$1 \Omega$, *-*] (3,2) 
		node[label={above:X}] {}
		to[R, l^=$2 \Omega$] (5.5,2)
		to[battery1, l=2 $V$] (5.5,0)
		-- (0,0)
		(3,2) to[C, l=1 ${\mu}F$] (3,0) 
		-- (3,-0.5) node[ground, label={right:G}] {};
	\end{circuitikz}
	\caption{}
	\label{fig:ckt-q1}
\end{figure}

\item Find $q_1$.\\
\solution\\
After infinite time with switch at P\\
The capacitor is charged\\
%Current in the ciruit $I = \frac{\sum(V)}{\sum(R)} = 1$ A\\
Applying KCL at X:
\begin{align}
	\dfrac{V_x - 1}{1} = \dfrac{2 - V_x}{2}\\
	\implies V_x = \dfrac{4}{3} \text{ V}\\
	q_1 = CV = 1 \text{ $\mu$C}
\end{align}

\item Show that the Laplace transform of $u(t)$ is $\frac{1}{s}$ and find the ROC.\\
\solution
\begin{align}
	\mathcal{U}(s) = \int_{0}^{\infty}u(t)e^{-st}dt \\
	= \int_{0}^{0}\frac{1}{2}e^{-st}dt + \int_{0}^{\infty}e^{-st}dt 
	= \frac{1}{s}\\
	\text{R.O.C: } Re(s) > 0
	\label{eq:L-u}
\end{align}
\item Show that 
	\begin{align}
		e^{-at}u(t) &\system{L} \frac{1}{s+a}, \quad a > 0
	\end{align}
and find the ROC.
\solution
\begin{align}
	e^{-at}u(t) &\system{L} \int_{0}^{\infty}u(t)e^{-(s + a)t}dt \\
	&= \frac{1}{s + a}\\
		\text{R.O.C: } Re(s) > -a
	\label{eq:L-u-shift}
\end{align}
\item Now consider the following resistive circuit transformed from 
Fig. \ref{fig:ckt}
\begin{figure}[!ht]
	\centering
	\includegraphics[width=\columnwidth]{/media/darkwake/VIB2/EE3900/cktsig/figs/lap-ckt.jpg}
	\caption{}
	\label{fig:lap-ckt}
\end{figure}
where 
\begin{align}
	u(t) \system{L} V_1(s)
	\\
	2u(t) \system{L} V_2(s)
\end{align}
Find the voltage across the capacitor $V_{C_0}(s)$.\\
\solution
Applying KCL at X:
\begin{align}
	\dfrac{V_x - \frac{1}{s}}{R_1} + s(V C_0)= \dfrac{\frac{2}{s} - V_x}{R_2}\\
	V(s) = \frac{\frac{1}{R_1} + \frac{2}{R_2}}{s\brak{\frac{1}{R_1} + \frac{1}{R_2} + sC_0}} \\
	= \frac{2R_1 + R_2}{R_1 + R_2}\brak{\frac{1}{s} - \frac{1}{\frac{1}{C_0}\brak{\frac{1}{R_1} + \frac{1}{R_2}} + s}} 
	\label{eq:V-s}
\end{align}
\item Find $v_{C_0}(t)$.  Plot using python.
\begin{align}
	& v_{C_0}(t) = \frac{2R_1 + R_2}{R_1 + R_2}u(t)\brak{1 - e^{-\brak{\frac{1}{R_1} + \frac{1}{R_2}}\frac{t}{C_0}}} \\
	&v_{C_0}(t) = \frac{4}{3}\brak{1 - e^{-\brak{1.5 \times 10^6}t}}u(t)
\end{align}
%The below code plots the graph \ref{fig:v1-t}
%\begin{lstlisting}
	%https://github.com/Vishwanath-123/EE3900/blob/main/EE3900-2022-main/cktsig/codes/2_6.py
%\end{lstlisting}
\item Verify your result using ngspice.\\
\solution
\begin{figure}[!htb]
\includegraphics[width=\columnwidth]{/media/darkwake/VIB2/EE3900/cktsig/figs/e2.6.png}
\caption{$v_{C_0}(t)$ before the switch is flipped}
\label{fig:v1-t}
\end{figure}
\vspace{3cm}
\item Obtain Fig. 
\ref{fig:lap-ckt}
using the equivalent differential equation.
\end{enumerate}

\section{Initial Conditions}
\begin{enumerate}[label=\arabic*.,ref=\thesection.\theenumi]
\numberwithin{equation}{section}
\item Find $q_2$ in Fig. \ref{fig:ckt}.\\
\solution
The circuit at steady state when the switch is at Q:\\
\begin{figure}[!htb]
	\begin{center}
		\begin{circuitikz} \draw
			(0,0) -- (0,2)
			node[label={above:Q}] {}
			to[R, l^=$1 \Omega$, *-*] (3,2) 
			node[label={above:X}] {}
			to[R, l^=$2 \Omega$] (5.5,2)
			to[battery1, l=2 $V$] (5.5,0)
			-- (0,0)
			(3,2) to[C, l=1 ${\mu}F$] (3,0) 
			-- (3,-0.5) node[ground, label={right:G}] {};
		\end{circuitikz}
	\end{center}
	\caption{}
	\label{fig:ckt-q2}
\end{figure}
At steady state: Capacitor is charged\\
Applying KCL at X.
\begin{align}
	\frac{V - 0}{1} + \frac{V - 2}{2} = 0\\
	\implies V = \SI[parse-numbers=false]{\frac{2}{3}}{\V}
	q_2 = \SI[parse-numbers=false]{\frac{2}{3}}{\micro\coulomb}
\end{align}                                         

\item Draw the equivalent $s$-domain resistive circuit when S is switched to position Q.  Use variables $R_1, R_2, C_0$ for the passive elements.
Use latex-tikz.\\
\label{prob:init}
\solution\\
\begin{figure}[!htb]
	\begin{center}
		\begin{circuitikz} 
			\ctikzset{resistor = european}
			\draw
			(0,0) -- (0,3)
			node[label={above:Q}] {}
			to[R, l^=$R_1$, *-*] (3,3) 
			node[label={above:X}] {}
			to[R, l^=$R_2$] (5.5,3)
			to[battery1, l= $\frac{2}{s} V$] (5.5,0)
			-- (0,0)
			(3,3) to[battery1, l=$\frac{4}{3s} V$] (3,2) to[R, l=$\frac{1}{sC_0}$] (3,0) 
			-- (3,-0.5) node[ground, label={right:G}] {};
		\end{circuitikz}
	\end{center}
	\caption{}
	\label{fig:sckt-q2}
\end{figure}
\newline

\item $V_{C_0}(s)$ = ?
\solution\\
Applying KCL at node X in Fig. \ref{fig:sckt-q2}
\begin{align}
	\frac{V - 0}{R_1} + \frac{V - \frac{2}{s}}{R_2} + sC_0\brak{V - \frac{4}{3s}} = 0 \\
	\implies V_{C_0}(s) = \frac{\frac{2}{sR_2} + \frac{4C_0}{3}}{\frac{1}{R_1} + \frac{2}{R_2} + sC_0}\\
	\label{eq:v2-s}
	V_{C_0}(s) = \frac{4}{3}\brak{\frac{1}{\frac{1}{C_0}\brak{\frac{1}{R_1} + \frac{1}{R_2}}+s}} \nonumber \\
	+ \frac{2}{R_2\brak{\frac{1}{R_1} +\frac{1}{R_2}}}\brak{\frac{1}{s} - \frac{1}{\frac{1}{C_0}\brak{\frac{1}{R_1} + \frac{1}{R_2}} + s}}
\end{align}
\item $v_{C_0}(t)$ = ? Plot using python.
Taking an inverse Laplace Transform,
\begin{align}
	&v_{C_0}(t) = \frac{4}{3}e^{-\brak{\frac{1}{R_1} + \frac{1}{R_2}}\frac{t}{C_0}}u(t) \nonumber \\ 
	&+ \frac{2}{R_2\brak{\frac{1}{R_1}+\frac{1}{R_2}}}\brak{1 - e^{-\brak{\frac{1}{R_1} + \frac{1}{R_2}}\frac{t}{C_0}}}u(t)
\end{align}
Substituting values gives
\begin{align}
	v_{C_0}(t) = \frac{2}{3}\brak{1 +e^{-\brak{1.5 \times 10^6}t}}u(t)
	\label{eq:v2-t}
\end{align}
The python code used to plot Fig. \ref{fig:v2-t}
\begin{lstlisting}
https://github.com/DarkWake9/EE3900/blob/main/cktsig/codes/e3.4.py
\end{lstlisting}
\item Verify your result using ngspice.\\
\solution
\begin{figure}[!htb]
	\includegraphics[width=\columnwidth]{/media/darkwake/VIB2/EE3900/cktsig/figs/e3.4.jpg}
	\caption{$v_{C_0}(t)$ after the switch is flipped}
	\label{fig:v2-t}
\end{figure}
 The following ngspice script simulates the given circuit
\begin{lstlisting}
https://github.com/DarkWake9/EE3900/blob/main/cktsig/codes/e3.cir
\end{lstlisting}
\item Find $v_{C_0}(0-), v_{C_0}(0+)$ and  $v_{C_0}(\infty) $.\\
\solution
\begin{align}
	v_{C_0}(0-) = \lim_{t \to 0-}v_{C_0}(t) = \frac{q_1}{C} = \frac{4}{3}\text{ V}\\
	v_{C_0}(0+) = \lim_{t \to 0+}v_{C_0}(t) = \SI[parse-numbers=false]{\frac{4}{3}}{\V}
\end{align}
\begin{align}
	v_{C_0}(\infty) = \lim_{t \to \infty}v_{C_0}(t) = \SI[parse-numbers=false]{\frac{2}{3}}{\V}
\end{align}

\item Obtain the Fig.  in problem 
\ref{prob:init}
using the equivalent differential equation.\\
\solution The equivalent circuit in the $t$-domain is shown below.

\begin{figure}[!htb]
	\begin{center}
		\begin{circuitikz} 
			\draw
			(0,0) -- (0,3)
			node[label={above:Q}] {}
			to[R, l^=$R_1$, *-*, i = $i_1$] (3,3) 
			node[label={above:X}] {}
			to[R, l^=$R_2$, i = $i_3$] (5.5,3)
			to[battery1, l= $\SI{2}{\V}$] (5.5,0)
			-- (0,0)
			(3,3) to[battery1, l=$\frac{4}{3} V$] (3,2) to[C, l=$C_0$, i = $i_2$] (3,0) ;
		\end{circuitikz}
	\end{center}
	\caption{}
	\label{fig:tckt-q2}
\end{figure}
From KCL and KVL,
\begin{align}
	&i_1 = i_2 +i_3 \\
	&i_1R_1 + \frac{4}{3} + \frac{1}{C_0}\int_{0}^{t}i_2dt = 0 \\
	&\frac{4}{3} + \frac{1}{C_0}\int_{0}^{t}i_2dt - i_3R_2 - 2 = 0 \\
\end{align}
Taking Laplace Transforms on both sides and using the properties of Laplace Transforms,
\begin{align}
	\text{Let }i(t) \system{L} I(s)\\
	\implies	I_1 = I_2 +I_3 \label{eq:s1}\\
	I_1R_1 + \frac{4}{3} + \frac{1}{sC_0}I_2 = 0 \\
	\frac{4}{3} + \frac{1}{sC_0}I_2 - I_3R_2 - 2 = 0 \label{eq:s3}
\end{align}
%Note that the capacitor is equivalent to a resistive element of resistance $R_C = \frac{1}{sC_0}$ in the $s$-domain. Equations \eqref{eq:s1} - \eqref{eq:s3} precisely describe Fig. \ref{fig:sckt-q2}. 


\end{enumerate}
\section{Bilinear Transform}
\begin{enumerate}[label=\arabic*.,ref=\thesection.\theenumi]
\numberwithin{equation}{section}
\item In Fig. 
\ref{fig:ckt},
consider the case when $S$ is switched to $Q$ right in the beginning. Formulate the differential equation.\\

\solution The equivalent circuit in the $t$-domain is shown below.

\begin{figure}[!htb]
	\begin{center}
		\begin{circuitikz} 
			\draw
			(0,0) -- (0,3)
			node[label={above:Q}] {}
			to[R, l^=$R_1$, *-*, i = $i_1$] (3,3) 
			node[label={above:X}] {}
			to[R, l^=$R_2$, i = $i_3$] (5.5,3)
			to[battery1, l= $V_2$] (5.5,0)
			-- (0,0)
			(3,3) to[C, l=$C_0$, i = $i_2$] (3,0) ;
		\end{circuitikz}
	\end{center}
	\caption{}
	\label{fig:tckt-q4}
\end{figure}

Applying KCL and KVL,
\begin{align}
	&i_1 = i_2 + i_3 \\
	&i_1R_1 + \frac{1}{C_0}\int_0^ti_2\, dt = 0 \\
	&i_3R_2 + 2 - \frac{1}{C_0}\int_0^ti_2\, dt = 0
\end{align}
Differentiating the above equations,
\begin{align}
	&\diff{i_1}{t} = \diff{i_2}{t} + \diff{i_3}{t} \label{eq:diff1}\\
	&R_1\diff{i_1}{t} + \frac{i_2}{C_0} = 0 \label{eq:diff2}\\
	&R_2\diff{i_3}{t} - \frac{i_2}{C_0} = 0 
	\label{eq:diff3}
\end{align}
Using \eqref{eq:diff1} and \eqref{eq:diff3} in \eqref{eq:diff2},
\begin{align}
	&R_1\brak{\diff{i_2}{t} + \diff{i_3}{t}} + \frac{i_2}{C_0} = 0 \\
	&R_1\diff{i_2}{t} + \brak{1 + \frac{R_1}{R_2}}\frac{i_2}{C_0} = 0 \\
	&\diff{i_2}{t} + \brak{\frac{1}{R_1} + \frac{1}{R_2}}\frac{i_2}{C_0} = 0 \\
	&\diff{i_2}{t} + \frac{i_2}{\tau} = 0
	\label{eq:diff-eqn-init}
\end{align}
where $\tau = \frac{C_0R_1R_2}{R_1 + R_2}$ is the RC time 
constant of the circuit. Note that $i_2(0) = \frac{V_2}{R_2}$ A and 
$i_2 = C_0\diff{V}{t}$, where $V$ is the voltage of the capacitor. 
Hence, integrating \eqref{eq:diff-eqn-init},
\begin{align}
	C_0\diff{V}{t} - \frac{V_2}{R_2} + \frac{C_0V}{\tau} &= 0 \\
	\implies \diff{V}{t} + \frac{V}{\tau} = \frac{V_2}{C_0R_2}
	\label{eq:diff-eqn}
\end{align}
\item Find $H(s)$ considering the ouput voltage at the capacitor.\\
\solution\\
\begin{align}
		H(s) = \frac{V_{C_0}(s)}{V_2(s)}
\end{align}
In the $s$ domain:
\begin{align}
	\frac{V_{C_0}}{R_1} + \frac{V_{C_0}}{\frac{1}{sC_0}} + \frac{V_{C_0} - V_2}{R_2} = 0 \\
	H(s)\brak{\frac{1}{R_1} + \frac{1}{R_2} + sC_0} = \frac{1}{R_2} \\
	H(s) = \frac{\frac{1}{R_2}}{\frac{1}{R_1} + \frac{1}{R_2} + sC_0} = \dfrac{1}{3 + sC_0}
	\label{eq:Hs}
\end{align}
\item Plot $H(s)$.  What kind of filter is it?\\
\begin{lstlisting}
\end{lstlisting}
\begin{figure}[!ht]
	\includegraphics[width=\columnwidth]{/media/darkwake/VIB2/EE3900/cktsig/figs/e4.3.jpg}
	\caption{Plot of $H(s)$.}
	\label{fig:Hs}
\end{figure}
It is a Low-Pass filter\\
\item Using trapezoidal rule for integration, formulate the difference equation
by considering 
\begin{align}
	y(n) = y(t)\vert_{t=n}
\end{align}
\solution\\
\item Find $H(z)$.
\item How can you obtain $H(z)$ from $H(s)$?
\solution\\
Apply a Bilinear Transform
\begin{align}
	s \rightarrow \dfrac{2}{T}\dfrac{z-1}{z+1}
\end{align}
\end{enumerate}

\end{document}